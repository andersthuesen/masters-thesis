\chapter{Methods}

\section{Tracking \& Matching}
We track the predicted staff and patients across multiple frames, indicated by $t = 1 \ldots T$, by iteratively assigning the predictions, $\{ P_i^{(t)} \}_{i=1}^{M_t}$, to the latest known tracks, $\{ Q^{(t)}_{j} \}_{j=1}^{N_t}$ by greedily picking the assignment with the minimum cost:
\begin{equation}
    \argmin_{i,j} \mathcal{L}_\text{track}(P^{(t)}_i, Q^{(t-1)}_j),
\end{equation}
until either all predictions or latest tracks have been assigned exactly once. In the case of any unassigned predictions, a new track is initialized. After tracks have been assigned, the latest track $Q^{(t)}_j$ is updated with the assigned predictions. We choose the following composite loss function:
\begin{equation}
    \mathcal{L}_\text{track}(P, Q) = \alpha \norm{P_\text{3D kpts} - Q_\text{3D kpts}}{2} + \beta \norm{P_\text{class} - Q_\text{class}}{\infty},
\end{equation}
incorporating both the Euclidean distance between the 3D joint keypoints as well as the predicted person class (staff/patient), with $\alpha$ and $\beta$ weighting the influence of each term. The tracks, $T_i$, are then greedily matched to the ground truth annotations $G_j$, minimizing the loss:
\begin{equation}
    \mathcal{L}_\text{match}(T, G) = \sum_t^{T} \begin{cases}
        \norm{a^{(t)}_{\text{track}, \text{2D kpts}} - b^{(t)}_{\text{track}, \text{2D kpts}}}{2} & \text{if}\ a^{(t)}_\text{track} \in a_\text{track} \\
        \gamma & \text{otherwise}
    \end{cases}
\end{equation}
over the trajectory time horizon $t = 1,2,\ldots,T$ where $\gamma$ is the punishment for not detecting the person.

\section{Human pose sequence optimization}
% Something about using the original poses + vPoser (human pose prior) + annotated keypoints.


\section{3D scene reconstruction}




% Two approaches 

% First approach:
% First uses the relative depth disparity maps and the rendered SMPL poses in order to restore the 
% metric depth scale and offset.
% Pros:
% - Relative depth maps are often more accurate and easier for a model to learn.
% Cons:
% - The process of restoring scale and offset is not very robust and can be prone to errors if the SMPL poses don't overlap precisely with the predicted depths.  
% - The reliance of poses at different distances in order to optimally regress the scale and offset parameters.

% Second approach:
% Using metric depth estimation
% Pros:
% - Doesn't rely on rendering SMPL poses and restoring scale and offset.
% - Domain is contained to indoor hospital and carehome rooms.
% Cons:
% - Less accurate predictions


\subsection*{Restoring depth scale and offsets}
We utilize pseudo ground truth disparity maps generated by the Depth Anything model, inversely proportional to the scene depth, in order to reconstruct a point cloud of the scene. To recover the depth scale and offset we rasterize the predicted SMPL poses, extract the z-buffer, compute the inverse depth and regress the intersection with the normalized disparity maps using the Random Sample Consensus (RANSAC) algorithm robust to outliers. 

\begin{equation}
    \begin{bmatrix}p
        x \\ y \\ 1/z \\ 1
    \end{bmatrix} = \begin{bmatrix}
        f_x & 0 & 0 & p_x \\
        0 & f_y & 0 & p_y \\
        0 & 0 & 1 & 0 \\
        0 & 0 & 0 & 1
    \end{bmatrix} \begin{bmatrix}
        X \\ Y \\ Z \\ 1
    \end{bmatrix}
\end{equation}


\begin{equation}
    X = \frac{Z}{f_x} \left( x - p_x \right) , \quad Y = \frac{Z}{f_x} \left(y - p_y \right)
\end{equation}
where $f_x$, $f_y$ are the horizontal and vertical focal lenghts respectively and ($p_x$, $p_y$) is the principal point often chosen as the center $(w/2,h/2)$ of the image. 

\section{Motion and scene representation}
Previous work on single human motion generation uses a canonical representation 

We use a 6D continuous representation of the joint angles as according to \cite{Zhou_2019_CVPR}.
% We represent the motion similar to 


\section{Diffusion model}



